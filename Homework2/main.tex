\documentclass[onecolumn, draftclsnofoot,10pt, compsoc]{IEEEtran}
\usepackage{graphicx}
\usepackage{url}
\usepackage{setspace}
\usepackage{titling}
\usepackage{listings}
\usepackage{hyperref}

\usepackage{geometry}
\geometry{textheight=9.5in, textwidth=7in}

\renewcommand{\lstlistingname}{Code}

\title{Homework Assignment 2\\\large CS444 Spring 2018}
\author{Brennan Douglas, Cooper Hutchins, Nicholas Giles}
\date{May 2, 2018}


\begin{document}
\begin{titlingpage}
			\maketitle
      \begin{abstract}
			\noindent Homework 2 required the design an implementation of an I/O scheduler in the Linux kernel.  Specifically, either a Look or C-Look algorithm.
      \end{abstract}
\end{titlingpage}

\newpage
\pagenumbering{arabic}

\clearpage
\singlespace


% Document body
\section{The design you plan to use to implement the LOOK algorithms.}

Our plan was to take advantage of the Linux linked list being doubly-linked and circular. So we plan to make the head of the queue to always be where the read/write head is and in order to find the head we search through the sort queue for when the jump from a high to low value happens, and once the head is found the new request will be added through an insertion sort into the necessary spot based on the location it wants to access. For dispatching, we plan on dispatching from the current head of the list, and then add it to the tail of the dispatch queue.  This way the order that was already created is persevered and the saw-tooth ordered requests are obtained.

\section{Questions:}
\subsection{What do you think the main point of this assignment is?}

To familiarize ourselves with using Linux kernel structs, implementations of basic schedulers, using Linux kernel libraries such as “linux/elevator.h”, and lastly to further cement our understanding of building, compiling, and running a virtual machine.

\subsection{How did you personally approach the problem? Design decisions, algorithm, etc.}

We chose to implement the C-Look algorithm after considering the design rigours required to implement either of the two algorithms and deciding it would be easier to implement an elevator which only moves in one direction rather than having to worry about splitting, sorting, and merging a request queue.

\subsection{How did you ensure your solution was correct? Testing details, for instance.}

Luckily for us, we did not have any runtime errors in our code, only compile errors. So our use of gdb was minimal. For the most part we tested our C-Look implementation by utilizing printk. Every request that was sorted would be printed out with the location of where it was sorted, and every request dispatched would be printed out with the block sectors it was dispatched to. We noticed that the printed dispatches showed upward movement, and then it would reset back to a lower number and continue the trend like a C-Look implementation should.

\subsection{What did you learn?}

We learned a lot of things. For example, we learned what the term elevators means to order I/O calls to the hard drive by sector and location in an ascending or descending order. We learned about block I/O scheduling and the importance of efficiency when scheduling the hard drive to read/write. Since the read write head of the hard drive needs to physically move around the hard drive platters, the use of elevators becomes efficient since it limits the amount of unnecessary movement by the read/write head. We learned about how Linux implements linked lists as doubly-linked circular linked lists, and more importantly how they are used to create scheduling queues. Moreover, we learned about what the C-Look and Look algorithms are. The Look algorithm is an elevator sorting scheduler that schedules read/write commands for whatever direction the read/write head is moving, so if the head is moving up the platter than the queue is ordered in an ascending manner, whereas if the head is moving down the queue is ordered in a descending manner. Like the Look algorithm, the C-Look algorithm is also an elevator sorting scheduler, but unlike Look, C-Look only schedules the queue for a single direction, and when the head reaches the end of that direction it quickly moves back down and starts from the next sector.

\subsection{How should the TA evaluate your work? Provide detailed steps to prove correctness.}

\begin{itemize}
		\item Apply the kernel patch file, "hw2.patch", and build the kernel.
		\item The build process should just run as the configuration file is part of the patch file.  However, it may ask you to specify which I/O Scheduler to use, if so select "CLOOK" and continue building.
		\item Source the correct environment file for your shell, we used the bash variant.
		\item Launch qemu using the following command (from the parent folder of the git repo):\\
		qemu-system-i386 -gdb tcp::5532 -S -nographic -kernel ./linux-yocto-3.19/arch/x86/boot/bzImage -drive file=core-image-lsb-sdk-qemux86.ext4,if=ide -enable-kvm -usb -localtime --no-reboot --append "root=/dev/hda rw console=ttyS0 debug"
		\item Then use gdb to continue the kernel, you should begin to see several "CLOOK" print statements as confirmation that the new scheduler is being used.
		\item Login and then run the "\textit{dmesg -n 7}" command to disable to "CLOOK" messages in the shell.
		\item From the root directory, "\textit{cd /}" then run the following command for a short while and then press \textit{Ctrl-C} to cancel it: "\textit{mkdir ./cpy \&\& cp -r ./* ./cpy}"
		\item Run "\textit{less +G /var/log/debug}" and view the "CLOOK" messages that were created from the previous command.
		\item There will be consecutive groups of add requests, and dispatch statements.  The dispatch statements will always be in an ascending order given a short enough period of time.  However, they may not come directly after the add requests and thus certain requests may be in the queue for an extended period of time.
\end{itemize}

\section{Git Log}

This is our git log for the work completed on the kernel.  This doubles as our work log.

\begin{center}
	\begin{tabular}{ |c|c|c|p{10cm}| }
		\hline
		\textbf{Detail} & \textbf{Date} & \textbf{Author} & \textbf{Description}\\\hline
        \href{git://git.yoctoproject.org/linux-yocto-3.19/commit/77bdaae3462de04df412ac2bca4e493bb5f98a55}{77bdaae} & 2018-04-25 17:28:55 & Brennan Douglas & added config file to git repo\\\hline
        \href{git://git.yoctoproject.org/linux-yocto-3.19/commit/34bc8f9f605ad4b0de2e16c4f78a9bc8a64cfd5e}{34bc8f9} & 2018-04-25 17:32:01 & Brennan Douglas & added back all.config to .gitignore\\\hline
        \href{git://git.yoctoproject.org/linux-yocto-3.19/commit/16eaede969c0657cdfb8232c191056a4b6609d4b}{16eaede} & 2018-04-25 17:35:54 & Brennan Douglas & updated .config local version tag to reflect hw2 branch\\\hline
        \href{git://git.yoctoproject.org/linux-yocto-3.19/commit/89292cade8d4661b075929749687b30878b047e8}{89292ca} & 2018-04-25 21:10:53 & Brennan Douglas & started clook implmentation file\\\hline
        \href{git://git.yoctoproject.org/linux-yocto-3.19/commit/a98e1c6d5efe859e1abe454fef6b6962087cff55}{a98e1c6} & 2018-04-25 21:35:50 & Brennan Douglas & added print statments\\\hline
        \href{git://git.yoctoproject.org/linux-yocto-3.19/commit/40d2c643941823903049adb0c51020c8b9ab23cf}{40d2c64} & 2018-04-25 21:43:12 & Nicholas Giles & Finished working on the clook{\textunderscore}add{\textunderscore}request\\\hline
        \href{git://git.yoctoproject.org/linux-yocto-3.19/commit/f2d8cba32210643d0fb3ddc80b1f52027ad99cad}{f2d8cba} & 2018-04-25 21:44:04 & Nicholas Giles & Merge branch 'hw2' of /scratch/spring2018/32/linux-yocto-3.19 into hw2\\\hline
        \href{git://git.yoctoproject.org/linux-yocto-3.19/commit/c3d85224bae8816c4c38a8c2ac8771de76608a93}{c3d8522} & 2018-04-25 21:45:22 & Nicholas Giles & Handled the merge conflict\\\hline
        \href{git://git.yoctoproject.org/linux-yocto-3.19/commit/550a566e3a12851b244fbb9b8d694cf13a23ef0a}{550a566} & 2018-04-25 21:48:41 & Brennan Douglas & updated configuration and building files\\\hline
        \href{git://git.yoctoproject.org/linux-yocto-3.19/commit/8c89b727efeb64c7ca8d0e1f4e8eb517a5a35fdc}{8c89b72} & 2018-04-25 21:58:46 & Brennan Douglas & Merge branch 'hw2' of /scratch/spring2018/32/linux-yocto-3.19 into hw2\\\hline
        \href{git://git.yoctoproject.org/linux-yocto-3.19/commit/44f57c19197da390373b4e7a35e998b079f96bed}{44f57c1} & 2018-04-25 22:26:29 & Nicholas Giles & Updated name of the file from clook{\textunderscore}sched.c to sstf{\textunderscore}sched.c and updated the Makefile for the change\\\hline
        \href{git://git.yoctoproject.org/linux-yocto-3.19/commit/63564c98f39c242f9f809ac580805488ee4ff1de}{63564c9} & 2018-04-25 22:30:43 & Brennan Douglas & updated config to add clook option\\\hline
        \href{git://git.yoctoproject.org/linux-yocto-3.19/commit/c5115d3b8ef68a1c11d4dfa3bf153a2b492f14af}{c5115d3} & 2018-04-25 22:31:05 & Nicholas Giles & Merge branch 'hw2' of /scratch/spring2018/32/linux-yocto-3.19 into hw2\\\hline
        \href{git://git.yoctoproject.org/linux-yocto-3.19/commit/fa13e4783ca9c01e47ce00261f3b80baebd0d3d0}{fa13e47} & 2018-05-01 07:52:57 & Brennan Douglas & updated add and dispatch function\\\hline
        \href{git://git.yoctoproject.org/linux-yocto-3.19/commit/2f0d199bb86f46305d377f8746f997fd517b8d4a}{2f0d199} & 2018-05-02 19:33:44 & Brennan Douglas & updated add algorithm\\\hline
	\end{tabular}
\end{center}

\end{document}
