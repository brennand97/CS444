\documentclass[onecolumn, draftclsnofoot,10pt, compsoc]{IEEEtran}
\usepackage{graphicx}
\usepackage{url}
\usepackage{setspace}
\usepackage{titling}
\usepackage{listings}

\usepackage{geometry}
\geometry{textheight=9.5in, textwidth=7in}

\renewcommand{\lstlistingname}{Code}

\title{Homework Assignment 1\\\large CS444 Spring 2018}
\author{Brennan Douglas, Cooper Hutchins, Nicholas Giles}
\date{\today}


\begin{document}
\begin{titlingpage}
			\maketitle
      \begin{abstract}
			\noindent Homework 1 requires the building and running of the yoco linux kernal in a qemu vm.  Then the solving of a concurrency problem using C and hardware level random number access.
      \end{abstract}
\end{titlingpage}

\newpage
\pagenumbering{arabic}

\clearpage
\singlespace


% Document body

\section{Command List}

This is the command list for setting up our directory on the server, building the kernel, and running the QEMU VM.

\begin{enumerate}
    \item mkdir /scratch/spring2018/32
    \item cd /scratch/spring2018/32
    \item cp /scratch/opt/poky/1.8/environment-setup-i586-poky-linux .
    \item cp /scratch/files/bzImage-qemux86.bin .
    \item cp /scratch/files/core-image-lsb-sdk-qemux86.ext4 .
    \item source environment-setup-i586-poky-linux
    \item git clone --branch v3.19.2 git://git.yoctoproject.org/linux-yocto-3.19
    \item cd linux-yocto-3.19/
    \item git checkout -b os2
    \item cp /scratch/files/config-3.19.2-yocto-standard .config
    \item make menuconfig \newline \indent Local Version (under General Settings) was changed to '-groupg32-hw1'.
    \item make -j4 all
    \item cd ..
    \item qemu-system-i386 -gdb tcp::5532 -S -nographic -kernel linux-yocto-3.19/arch/x86/boot/bzImage -drive file=core-image-lsb-sdk-qemux86.ext4,if=virtio -enable-kvm -net none -usb -localtime --no-reboot --append "root=/dev/vda rw console=ttyS0 debug"
    
    \vspace{3mm}
    \emph{In new terminal:}
    
    \item gdb -tui
    \item target remote:5532
    \item continue
    
    \vspace{3mm}
    \emph{In pervious terminal:}
    
    \item root \newline This username was entered when prompted.
    \item uname -a
    \item reboot
\end{enumerate}


% References
%\cite[Sec 3.8]{freebsd}
%\bibliographystyle{plain}
\bibliographystyle{IEEEtran}
\bibliography{./ref}

\end{document}
